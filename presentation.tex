\documentclass[11pt,xcolor={dvipsnames},hyperref={pdftex,pdfpagemode=UseNone,hidelinks,pdfdisplaydoctitle=true},usepdftitle=false]{beamer}
\usepackage{presentation}
% Enter title of presentation PDF:
\hypersetup{pdftitle={Une Planète, Un Foyer : Résumé}}
% Enter link to PDF file with figures:
%\newcommand{\pdf}{figures.pdf}

\begin{document}
% Enter presentation title:
\title{Une Planète, Un Foyer : Résumé}
% Enter presentation information:
\information%
% Enter link to research paper (optional; comment line if not needed):
% Replace with GitHub link of the presentation
[https://www.bic.org/publications/one-planet-one-habitation]%
% Enter presentation authors:
{Communauté internationale bahá’íe}%
% Enter presentation location and date (optional; comment line if not needed):
{Stockholm+50 -- 1 juin 2022}
\frame{\titlepage}

\begin{frame}
  \frametitle{Sources et références}
  \begin{itemize}
    \item \href{https://www.bic.org/publications/one-planet-one-habitation}%
      {Publications}:
      \begin{itemize}
        \item
          \href{https://www.bic.org/sites/default/files/pdf/opoh_magazine-french-online.pdf}%
          {Français, brochure en PDF}
        \item
          \href{https://www.bic.org/statements/one-planet-one-habitation-bahai-perspective-recasting-humanitys-relationship-natural-world}%
          {Anglais, texte}
        \item
          \href{https://www.bic.org/sites/default/files/pdf/one_planet_one_habitation.pdf}%
          {Anglais, brochure en PDF}
      \end{itemize}
    \item
      \href{https://opoh.bic.org}{Exemples d'efforts en matière de durabilité
      et d'environnement déployés par des communautés bahá'íes du monde.}
    \item Cette présentation est basée sur
      \href{https://iefworld.org/fl/OPOHsummary.pdf}%
      {un résumé, en anglais, créé par le Forum international de l'environnement}
    \item
      \href{https://iefworld.org/elcentre.htm}%
      {Matériel d'apprentissage en ligne du Forum international de l'environnement.}
    \item Adapté par M.\ Semple en français pour une discussion de groupe à
      Bulle, Suisse.
  \end{itemize}
\end{frame}

\begin{frame}[allowframebreaks=0.8]
  \frametitle{La Communauté internationale bahá'íe}
  \begin{itemize}
    \item La Communauté internationale bahá'íe a été créée en avril 1948 et
      reconnue comme une \al{ONG internationale} représentant les membres de la
      foi bahá'íe.
    \item Son \al{objectif} est de contribuer à la formation collective des
      attitudes nécessaires pour faire progresser la prospérité et la justice
      dans le monde.
    \item Elle contribue aux discours politiques au niveau international dans
      divers espaces sociaux où la pensée, l'opinion publique et la politique
      prennent forme et évoluent.
    \item Elle promeut et applique des principes dérivés des enseignements de la
      foi bahá'íe pour développer une civilisation unie et durable et contribuer
      à
    \begin{itemize}
      \item l'avancement des droits de l'homme
      \item la promotion des femmes
      \item l'éducation universelle
      \item l'encouragement d'un développement économique juste
      \item la protection de l'environnement
      \item un sens de la citoyenneté mondiale.
    \end{itemize}
  \end{itemize}
\end{frame}

\begin{frame}
  \frametitle{Stockholm+50}
  \begin{itemize}
    \item Réunion internationale convoquée par l'Assemblée générale des
      Nations unies à Stockholm en 2022
    \item Thème : « une planète saine pour la prospérité de tous »
    \item 50\textsuperscript{e} anniversaire de la conférence des Nations unies
      sur l'environnement de 1972, qui a fait de l'environnement une question
      mondiale urgente pour la première fois
  \end{itemize}
\end{frame}

\begin{frame}
  \frametitle{Structure de la déclaration}
  \begin{itemize}
    \item 8 titres (thèmes) pour 44 paragraphes de texte principal
    \item \alg{6 cases séparées avec du texte sur des thèmes particuliers}
    \item \alr{6 pistes de réflexion }
    \item courtes citations tirées des écrits bahá'ís
  \end{itemize}
\end{frame}

\begin{frame}[allowframebreaks=0.8]
  \frametitle{Principaux thèmes}
  Une Perspective Bahá’íe sur une Nouvelle Définition des Rapports de l’Humanité
  avec le Monde de la Nature.
  \begin{enumerate}
    \item Le monde de la nature 
      \begin{itemize}
        \item \alg{La tutelle du monde de la nature}
      \end{itemize}
    \item Un seul peuple dans une seule patrie mondiale \alr{+}
      \begin{itemize}
        \item \alg{Habiliter les protagonistes du changement transformateur}
      \end{itemize}
    \item Le consensus dans l'action \alr{+}
    \item Redéfinir le progrès \alr{+}
      \begin{itemize}
        \item \alg{Repenser les accords économiques}
      \end{itemize}
    \item S’aligner sur des principes plus élevés \alr{+}
      \begin{itemize}
        \item \alg{La science et la religion : Des systèmes complémentaires de
          connaissances et de pratiques}
      \end{itemize}
    \item La justice comme processus et résultat \alr{+}
      \begin{itemize}
        \item \alg{L’apprentissage en tant que mode de fonctionnement}
      \end{itemize}
    \item Accepter le rôle de l’État \alr{+}
      \begin{itemize}
        \item \alg{Le lieu de la prise de décisions}
      \end{itemize}
    \item Le monde qui nous attend
  \end{enumerate}
\end{frame}

\begin{frame}
  \heading{Thèmes}
\end{frame}

\begin{frame}[allowframebreaks=0.8]
  \frametitle{Le monde de la nature}
  \begin{quote}
     Cette étendue de terre n’est qu’une seule patrie et une seule demeure. Il 
     vous appartient d’abandonner toute vaine gloire, source d’aliénation, et 
     de tourner vos cœurs vers tout ce qui garantit l’harmonie.

     \raggedleft -- Bahá'u'lláh
  \end{quote}
  \begin{itemize}
    \item Le monde naturel, dans toute sa splendeur, nous montre la
      signification de \al{l'interdépendance}. De la biosphère dans son ensemble au
      plus petit micro-organisme, il démontre à quel point une forme de vie est
      dépendante de nombreuses autres et comment les \al{déséquilibres} d'un système
      affectent l'ensemble interconnecté.
    \item L'humanité est dépendante de ce grand système, mais alors que
      l'espèce humaine n'a jamais eu autant de pouvoir pour façonner le monde
      physique à l'échelle planétaire, ce même pouvoir, lorsqu'il n'est pas
      considéré avec prudence et qu'il ignore le bien commun présent et futur,
      a des conséquences mondiales et potentiellement irréversibles.
    \item Alors que les graves conséquences du dépassement des limites
      planétaires deviennent de plus en plus évidentes, qu'il s'agisse du
      changement climatique, de la perte de biodiversité ou de la dégradation et
      de la pollution de l'environnement, l'humanité doit développer des
      relations plus collaboratives et constructives entre ses peuples et avec
      l'environnement naturel.
    \item Aujourd'hui, nous devons agir beaucoup plus rapidement et à plus
      grande échelle, en modifiant l'organisation et le fonctionnement des
      affaires humaines. La question qui se pose aux nations et aux dirigeants
      du monde est de savoir si les mesures nécessaires seront prises par choix
      conscient et par prévention, ou si elles seront causées par la
      destruction et la souffrance dues à l'effondrement de l'environnement.
  \end{itemize}
\end{frame}

\begin{frame}[allowframebreaks=0.8]
  \frametitle{Un peuple dans une patrie mondiale}
  \begin{quote}
    Y a-t-il une seule action au monde qui serait plus noble que de servir le
    bien commun ? \ldots\ Non, par le Seigneur Dieu !

    \raggedleft -- Écrits saints bahá’ís
  \end{quote}
  \begin{itemize}
    \item Vue de la planète entière, l'humanité est un seul peuple vivant dans
      une seule patrie mondiale. La conscience de cette unité, en appliquant la
      justice, est le seul fondement sur lequel des sociétés durables peuvent
      être créées.
    \item Chaque peuple, à sa manière, célèbre la beauté et l'abondance de la
      nature. Les traditions de chaque culture reconnaissent ce patrimoine
      inestimable qui répond à nos besoins physiques et spirituels. La
      construction d'un monde durable apportera l'unité à la fois dans l'effort
      partagé et dans la célébration joyeuse.
    \item L'unité de l'humanité comprend des variations d'expression, de culture
      ou d'organisation sociale, ce que nous appelons l'unité dans la diversité.
      Dans le monde naturel, les systèmes dépendent également de nombreuses
      espèces pour fonctionner et être résistants.
    \item Dans les affaires humaines, la diversité de pensée, d'origine et
      d'approche est tout aussi importante. La vérité naît de l'interaction de
      perspectives et d'expériences diverses. Trop de points de vue et
      d'opinions similaires peuvent entraîner des dangers et des ruptures.
    \item Les contributions de beaucoup plus de personnes sont nécessaires pour
      rééquilibrer notre relation avec le monde naturel. La présomption de
      supériorité d'un groupe par rapport à un autre, en fonction de la
      nationalité, de la race, de la richesse ou de toute autre caractéristique,
      empêche le consensus et l'action coordonnée, et sape la motivation à
      travailler pour le bien commun, qu'il soit social ou écologique.
  \end{itemize}
\end{frame}

\begin{frame}[allowframebreaks=0.8]
  \frametitle{Le consensus dans l'action}
  \begin{quote}
    Nous devons constamment établir de nouvelles bases pour le bonheur humain et
    créer et promouvoir de nouveaux instruments à cette fin.

    \raggedleft -- Écrits saints bahá’ís
  \end{quote}
  \begin{itemize}
    \item L'évolution de l'humanité vers une relation plus durable et plus
      harmonieuse avec le monde naturel nécessitera un accord solide et une
      volonté collective autour de principes clés pour les affaires de la
      communauté internationale, tels que la gestion, l'interdépendance et la
      justice.
    \item L'écart entre les paroles et les actes montre que les principes liés à
      la durabilité n'influencent pas encore les choix et les comportements des
      nations.
    \item Nous avons besoin d'actes, pas de mots. L'engagement en faveur de
      principes et de valeurs clés peut aider les sociétés à dépasser des
      intérêts limités ou égoïstes.
    \item L'action doit être cohérente avec les principes qui sont
      collectivement adoptés et défendus par tous. L'ordre international doit
      faciliter les réponses planétaires aux défis planétaires.
  \end{itemize}
\end{frame}

\begin{frame}[allowframebreaks=0.8]
  \frametitle{Redéfinir le progrès}
  \begin{quote}
    L’ajustement des conditions de vie doit être tel que la pauvreté
    disparaisse, que chacun, autant que possible \ldots\ ait sa part de confort
    et de bien-être.

    \raggedleft -- Écrits saints bahá’ís
  \end{quote}
  \begin{itemize}
    \item Pour améliorer la relation de l'humanité avec le monde naturel, nous
      devons redéfinir les notions de progrès, de civilisation et de
      développement. Quelles sont les qualités qui permettent de juger de la
      réussite d'une personne, d'une nation ou d'une entreprise ? Pour quelles
      raisons sont-ils félicités et appréciés ?
    \item Tant que nos valeurs donneront la priorité aux possessions sur les
      relations ou à l'acquisition sur la responsabilité, et que nous attendrons
      une croissance infinie sur une planète finie, un monde durable restera
      hors de portée. De telles valeurs affectent l'esprit humain et conduisent
      à l'excès, à l'exploitation et à l'épuisement, avec des extrêmes de
      richesse et de pauvreté. Le progrès doit être compris en termes nouveaux.
    \item Aucun pays n'est un exemple de développement durable. Nous pensions
      que le développement était synonyme d'industrialisation, de capacité
      technologique et de croissance macroéconomique, mais ces éléments laissent
      de nombreuses personnes insatisfaites et en difficulté, tandis que de
      nombreuses autres populations dans le monde sont confrontées à des
      injustices. Aucun mode de vie ni aucune vision de la société ne peut être
      pris comme modèle pour l'ensemble de l'humanité.
    \item La redéfinition du progrès exige une compréhension élargie de
      nous-mêmes en tant qu'espèce, y compris des vérités sur l'esprit humain
      lui-même. L'hypothèse matérialiste simpliste qui considère l'individu
      comme une unité économique purement intéressée, en concurrence avec
      d'autres pour accumuler une part toujours plus grande des ressources
      matérielles de la planète, sous-tend toujours l'ordre mondial.
    \item Une compréhension plus précise de la nature humaine inclurait des
      qualités telles que la confiance, le soutien mutuel, l'engagement envers
      la vérité et le sens des responsabilités, qui sont les éléments
      constitutifs d'un ordre social stable, garantissant que notre quête de
      prospérité englobe les nombreux autres aspects du bien-être individuel et
      collectif.
    \item La redéfinition du progrès pourrait inclure de nouvelles approches de
      la propriété et de l'utilisation, de nouvelles formes d'organisation
      urbaine, de nouvelles méthodes d'agriculture, de production d'énergie et
      de transport, et de vastes possibilités s'offrent à l'humanité.
  \end{itemize}
\end{frame}

\begin{frame}[allowframebreaks=0.8]
  \frametitle{S’aligner sur des principes plus élevés}
  \begin{quote}
    La religion et la science sont les deux ailes qui permettent à
    l’intelligence de l’homme de s’élever vers les hauteurs, et à l’âme humaine
    de progresser. Il n’est pas possible de voler avec une aile seulement.

    \raggedleft -- Écrits saints bahá’ís
  \end{quote}
  \begin{itemize}
    \item L'existence de l'humanité est régie non seulement par des forces
      physiques, mais aussi par des lois sociales et morales de cause à effet.
      La cupidité est intrinsèquement corrosive pour le bien commun, même si
      elle est habilement justifiée ou dissimulée. Les actes de compassion
      désintéressée ont invariablement le pouvoir de motiver et d'inspirer, même
      s'ils semblent simples ou isolés.
    \item La voie vers une relation plus harmonieuse avec la nature, au-delà de
      l'ajustement technologique, doit impliquer que les communautés et les
      sociétés apprennent à vivre selon des principes plus élevés.
    \item Les enseignements religieux peuvent libérer les qualités élevées
      latentes en chaque individu, en créant des communautés qui mettent
      activement en pratique les valeurs transcendantes pour l'amélioration de
      tous.
    \item « Le mérite de l'homme réside dans le service et la vertu, et non dans
      l'étalage de la richesse et de la fortune », déclare Bahá'u'lláh, un
      exemple de valeurs qui transcendent la seule prospérité matérielle et qui
      peuvent aider le mouvement environnemental et l'humanité dans son
      ensemble.
  \end{itemize}
\end{frame}

\begin{frame}[allowframebreaks=0.8]
  \frametitle{La justice comme processus et résultat}
  \begin{quote}
    La connaissance est comme des ailes pour la vie de l’homme et une échelle
    pour son ascension. Il incombe à chacun de l’acquérir.

    \raggedleft -- Écrits saints bahá’ís
  \end{quote}
  \begin{itemize}
    \item La justice est au cœur de l'unité au niveau planétaire. De profondes
      injustices sont commises à l'égard des personnes et de la planète dans la
      souffrance généralisée résultant de la relation extractive de l'humanité
      avec le monde naturel, lorsqu'un petit nombre bénéficie de l'utilisation
      excessive des ressources de la terre tout en nuisant à beaucoup d'autres,
      lorsque les désirs immédiats l'emportent sur les besoins fondamentaux des
      générations futures.
    \item Pour corriger ces maux, il faudra faire preuve d'honnêteté, de
      créativité, de persévérance et d'humilité. La prise de décision doit
      inclure les voix de ceux qui ont été désavantagés par l'ordre actuel, en
      s'appuyant sur les connaissances des populations et des peuples indigènes
      vivant en harmonie avec le monde naturel, et en créant des modèles plus
      holistiques et durables pour les générations actuelles et futures.
    \item La justice exige que les bénéfices de la civilisation humaine soient
      répartis équitablement et que la responsabilité d'entreprendre les
      transitions nécessaires reflète les contributions historiques à la crise
      climatique actuelle. Nous avons également besoin de processus justes. Au
      niveau individuel, la justice exige que chacun fasse preuve d'impartialité
      dans ses jugements et d'équité dans le traitement qu'il réserve aux
      autres. Au niveau du groupe, c'est la conscience que les intérêts de
      l'individu et ceux de la société sont étroitement liés. Elle exige
      également de rechercher la vérité bien au-delà des modèles actuels de
      négociation et de compromis, en utilisant un processus de consultation et
      de prise de décision fondé sur des principes, ouvert et basé sur des
      faits.
    \item À tous les niveaux, la capacité à manifester la justice - et
      l'engagement à le faire - doivent être renforcés. Des relations justes et
      équitables sont le fondement de tout mouvement mondial unifié pour le
      bien commun.
  \end{itemize}
\end{frame}

\begin{frame}[allowframebreaks=0.8]
  \frametitle{Accepter le rôle de l’État}
  \begin{quote}
     [Le principe de l’unicité de l’humanité] insiste sur la nécessité de
     subordonner les impulsions et les intérêts nationaux aux revendications
     impérieuses d’un monde unifié ; elle refuse une centralisation excessive,
     d’une part, et rejette toute tentative d’uniformité, de l’autre.

    \raggedleft -- Écrits saints bahá’ís
  \end{quote}
  \begin{itemize}
    \item Chacun peut jouer un rôle dans la construction d'un monde plus
      durable. Les communautés locales peuvent faire beaucoup pour l'action
      collective en utilisant les capacités d'innovation de leurs membres. Les
      jeunes font constamment preuve d'ouverture à de nouveaux modes
      d'organisation de la société, d'une volonté d'apprendre par l'action en
      première ligne et d'une volonté de s'engager dans des entreprises de haut
      niveau et pour le bien-être des générations futures.
  \end{itemize}
\end{frame}

\begin{frame}[allowframebreaks=0.8]
  \frametitle{Le monde qui nous attend}
  \begin{itemize}
    \item Notre vision est celle d'une civilisation mondiale florissante en harmonie avec l'environnement naturel. Ce monde est celui de l'intégration et de l'équilibre, de la beauté et de la maturité. C'est un monde où le sens du progrès est redéfini, où les communautés et les individus travaillent ensemble, avec le soutien des institutions, à la réalisation de leurs plus grands espoirs. C'est un monde sans les compromis moraux destructeurs - sociaux, économiques et environnementaux - trop souvent considérés comme nécessaires au progrès.
    \item Le mouvement vers cette vision a commencé, avec de grandes ambitions et des appels à l'action. Pourtant, la transformation est trop lente. Plus nous attendons, plus ce sera difficile. L'humanité se rendra-t-elle compte que son propre destin et celui de la planète sont irrévocablement liés ? Ou faudra-t-il des catastrophes encore plus grandes pour la pousser à agir ?
    \item Le fossé entre les paroles et les actes est un défi majeur. Pourtant, un consensus beaucoup plus fort et une volonté collective parmi les nations sont nécessaires autour des valeurs exigées par le stade actuel de développement de l'humanité, en mettant ces valeurs en pratique pour le bien commun et en rejetant tout ce qui s'y oppose. Il s'agit là d'une entreprise de grande envergure, qui laissera un héritage inestimable aux générations futures. Unissons-nous pour être à la hauteur de ses exigences.
  \end{itemize}
\end{frame}

\lastslide

\end{document}
